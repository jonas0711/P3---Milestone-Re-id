\chapter{Experiments} \label{cha: experiments}
In order to determine whether adding an SR module to the ReID model will add any notable value, several experiments are conducted to create a clear and fair method for comparison. In order to test if \ac{SR} is worth implementing in terms of adding extra parameters to the pipeline, it is first tested, if strong data augmentation can improve model performance in terms of generalization across different image scales and resolution. For that reason is both a baseline model and a model trained using strong augmentations firstly trained to determine the need for \ac{SR} implementation.

\section{OOD Test} \label{sec: oodtestny}
Since the implemented person \acs{ReID} model is trained on the Market-1501 dataset, a new OOD test must be conducted to fulfill this project's definition of an OOD test, as described in Section \ref{outofdistributiontest}. This OOD test is conducted using the dataset DukeMTMC-ReID under three different conditions, as well as the dataset iLIDS-VID which is unknown to OSNet. They are presented in Table \ref{tab:newood_conditions}.
\begin{table}[H]
\centering
\renewcommand{\arraystretch}{1.2}
\begin{tabular}{
|>{\raggedright\arraybackslash}p{0.28\linewidth}
|>{\raggedright\arraybackslash}p{0.50\linewidth}
|>{\centering\arraybackslash}p{0.20\linewidth}|
}
\hline
\multicolumn{3}{|c|}{\cellcolor[HTML]{D8E9F7}\textbf{Dataset for OOD Test}} \\
\hline
\rowcolor[HTML]{D8E9F7}
\textbf{Condition} & \textbf{Description} & \textbf{Dataset} \\
\hline
Unmodifed &
Unmodified images, used as the reference baseline. & DukeMTMC-
ReID \\
\hline
Downscale: \newline 0.75 / 0.5 / 0.25 &
Images downscaled with a scaler, using height and width of the image.  &
DukeMTMC-
ReID \\
\hline
JPEG Compression: \newline 50 / 25 / 15 &
Images compressed using JPEG at quality levels 50 (medium), 25 (low), and 15 (very low). &
DukeMTMC-
ReID \\
\hline
Unmodified &
Image pairs from an unseen dataset, with primarily indoor images, to test cross-domain generalization. &
iLIDS-VID \\
\hline
\end{tabular}
\caption{Overview of the datasets used to evaluate OSNet under degraded and out-of-distribution conditions.}
\label{tab:newood_conditions}
\end{table}
All models presented in this Chapter will be finally tested and compared using this test.

\section{Baseline OSNet} \label{sec: baselineosnet}
The OOD test conducted in Section \ref{sec: oodtestny} highlights that the off-the-shelf OSNet model fails when confronted with images that are heavily augmented and simulate the domain shift in real-world deployment. To ensure a fair comparison with the augmentation trained model, a baseline model is trained from scratch, using the three datasets presented in Chapter \ref{cha:data}, and tested on DukeMTMC-ReID and ILIDS-vid. The light training augmentation pipeline, presented in \ref{tab:three_subtables}, is used for training of the baseline model. 
\\\\
While other training setups have been tried, the final model is trained with a batch size of 64 and a learning rate of \(1e-3\). The learning rate is reduced by a factor of 0.1 at epochs 100, 175, and 225 using a multi-step scheduler.

The results of the OOD show, that the model fail XXXXX

\begin{table}[H]
\centering
\begin{tabular}{|
    >{\centering\arraybackslash}p{0.17\linewidth} |
    >{\centering\arraybackslash}p{0.17\linewidth} |
    >{\centering\arraybackslash}p{0.17\linewidth} |
    >{\centering\arraybackslash}p{0.17\linewidth} |
    >{\centering\arraybackslash}p{0.17\linewidth} |
    }
\hline
\multicolumn{5}{|c|}{\cellcolor[HTML]{D8E9F7}\textbf{Test on Different Scales}} \\\hline
\rowcolor[HTML]{D8E9F7}
{} & Scale 1& Scale 0.75& Scale 0.5&
Scale 0.25\\ \hline
\textbf{mAP}    & & & & \\ \hline
\textbf{Rank-1} & & & & \\ \hline
\textbf{Rank-5} & & & & \\ \hline
\textbf{Rank-10}& & & & \\ \hline
\textbf{Rank-20}& & & & \\ \hline
\end{tabular}
\caption{Compression test of OSNet on 25 pairs from Market-1501 dataset.}
\label{tab:compressiontest_result}
\end{table}

\begin{table}[H]
\centering
\begin{tabular}{|
    >{\centering\arraybackslash}p{0.17\linewidth} |
    >{\centering\arraybackslash}p{0.17\linewidth} |
    >{\centering\arraybackslash}p{0.17\linewidth} |
    >{\centering\arraybackslash}p{0.17\linewidth} |
    >{\centering\arraybackslash}p{0.17\linewidth} |
    }
\hline
\multicolumn{5}{|c|}{\cellcolor[HTML]{D8E9F7}\textbf{Test on Compression with Differing Quality Levels}} \\\hline
\rowcolor[HTML]{D8E9F7}
{} & 100 & Mild 75& Medium 50&
Heavy 25\\ \hline
\textbf{mAP}    & & & & \\ \hline
\textbf{Rank-1} & & & & \\ \hline
\textbf{Rank-5} & & & & \\ \hline
\textbf{Rank-10}& & & & \\ \hline
\textbf{Rank-20}& & & & \\ \hline
\end{tabular}
\caption{Compression test of OSNet on 25 pairs from Market-1501 dataset.}
\label{tab:compressiontest_result}
\end{table}

\begin{table}[H]
\centering
\begin{tabular}{|>{\centering\arraybackslash}
    p{0.17\linewidth} 
    |>{\centering\arraybackslash}p{0.17\linewidth} 
    |}
\hline
\multicolumn{2}{|c|}{\cellcolor[HTML]{D8E9F7}\textbf{Test on iLIDS-VID}}\\ \hline
\textbf{mAP}    & \\ \hline
\textbf{Rank-1} & \\ \hline
\textbf{Rank-5} & \\ \hline
\textbf{Rank-10}& \\ \hline
\textbf{Rank-20}& \\ \hline
\end{tabular}
\caption{Compression test of OSNet on 25 pairs from Market-1501 dataset.}
\label{tab:compressiontest_result}
\end{table}

\section{Robust OSNet} \label{sec: robustosnet}
This experiment explores if training the person \acs{ReID} model on heavily augmented data will produce a more robust model than the baseline presented in Section \ref{sec: baselineosnet}. The applied data augmentation is presented in Table \ref{tab:three_subtables}. The model is trained same way as the baseline model to ensure fair comparison.
\begin{table}[H]
\centering
\begin{tabular}{|
    >{\centering\arraybackslash}p{0.17\linewidth} |
    >{\centering\arraybackslash}p{0.17\linewidth} |
    >{\centering\arraybackslash}p{0.17\linewidth} |
    >{\centering\arraybackslash}p{0.17\linewidth} |
    >{\centering\arraybackslash}p{0.17\linewidth} |
    }
\hline
\multicolumn{5}{|c|}{\cellcolor[HTML]{D8E9F7}\textbf{Test on Different Scales}} \\\hline
\rowcolor[HTML]{D8E9F7}
{} & Scale 1& Scale 0.75& Scale 0.5&
Scale 0.25\\ \hline
\textbf{mAP}    & & & & \\ \hline
\textbf{Rank-1} & & & & \\ \hline
\textbf{Rank-5} & & & & \\ \hline
\textbf{Rank-10}& & & & \\ \hline
\textbf{Rank-20}& & & & \\ \hline
\end{tabular}
\caption{Compression test of OSNet on 25 pairs from Market-1501 dataset.}
\label{tab:compressiontest_result}
\end{table}

\begin{table}[H]
\centering
\begin{tabular}{|
    >{\centering\arraybackslash}p{0.17\linewidth} |
    >{\centering\arraybackslash}p{0.17\linewidth} |
    >{\centering\arraybackslash}p{0.17\linewidth} |
    >{\centering\arraybackslash}p{0.17\linewidth} |
    >{\centering\arraybackslash}p{0.17\linewidth} |
    }
\hline
\multicolumn{5}{|c|}{\cellcolor[HTML]{D8E9F7}\textbf{Test on Compression with Differing Quality Levels}} \\\hline
\rowcolor[HTML]{D8E9F7}
{} & 100 & Mild 75& Medium 50&
Heavy 25\\ \hline
\textbf{mAP}    & & & & \\ \hline
\textbf{Rank-1} & & & & \\ \hline
\textbf{Rank-5} & & & & \\ \hline
\textbf{Rank-10}& & & & \\ \hline
\textbf{Rank-20}& & & & \\ \hline
\end{tabular}
\caption{Compression test of OSNet on 25 pairs from Market-1501 dataset.}
\label{tab:compressiontest_result}
\end{table}

\begin{table}[H]
\centering
\begin{tabular}{|>{\centering\arraybackslash}
    p{0.17\linewidth} 
    |>{\centering\arraybackslash}p{0.17\linewidth} 
    |}
\hline
\multicolumn{2}{|c|}{\cellcolor[HTML]{D8E9F7}\textbf{Test on iLIDS-VID}}\\ \hline
\textbf{mAP}    & \\ \hline
\textbf{Rank-1} & \\ \hline
\textbf{Rank-5} & \\ \hline
\textbf{Rank-10}& \\ \hline
\textbf{Rank-20}& \\ \hline
\end{tabular}
\caption{Compression test of OSNet on 25 pairs from Market-1501 dataset.}
\label{tab:compressiontest_result}
\end{table}

\section{EDSR and Baseline OSNet} \label{sec: sr+baseline}
Based on the results presented in Section \ref{sec: robustosnet}, the problem of images being out of distribution is not adequately solved. Therefore, a second experiment is conducted. The baseline model presented in Section \ref{sec: baselineosnet} is combined with a chosen SR model. 
\begin{table}[H]
\centering
\begin{tabular}{|
    >{\centering\arraybackslash}p{0.17\linewidth} |
    >{\centering\arraybackslash}p{0.17\linewidth} |
    >{\centering\arraybackslash}p{0.17\linewidth} |
    >{\centering\arraybackslash}p{0.17\linewidth} |
    >{\centering\arraybackslash}p{0.17\linewidth} |
    }
\hline
\multicolumn{5}{|c|}{\cellcolor[HTML]{D8E9F7}\textbf{Test on Different Scales}} \\\hline
\rowcolor[HTML]{D8E9F7}
{} & Scale 1& Scale 0.75& Scale 0.5&
Scale 0.25\\ \hline
\textbf{mAP}    & & & & \\ \hline
\textbf{Rank-1} & & & & \\ \hline
\textbf{Rank-5} & & & & \\ \hline
\textbf{Rank-10}& & & & \\ \hline
\textbf{Rank-20}& & & & \\ \hline
\end{tabular}
\caption{Compression test of OSNet on 25 pairs from Market-1501 dataset.}
\label{tab:compressiontest_result}
\end{table}

\begin{table}[H]
\centering
\begin{tabular}{|
    >{\centering\arraybackslash}p{0.17\linewidth} |
    >{\centering\arraybackslash}p{0.17\linewidth} |
    >{\centering\arraybackslash}p{0.17\linewidth} |
    >{\centering\arraybackslash}p{0.17\linewidth} |
    >{\centering\arraybackslash}p{0.17\linewidth} |
    }
\hline
\multicolumn{5}{|c|}{\cellcolor[HTML]{D8E9F7}\textbf{Test on Compression with Differing Quality Levels}} \\\hline
\rowcolor[HTML]{D8E9F7}
{} & 100 & Mild 75& Medium 50&
Heavy 25\\ \hline
\textbf{mAP}    & & & & \\ \hline
\textbf{Rank-1} & & & & \\ \hline
\textbf{Rank-5} & & & & \\ \hline
\textbf{Rank-10}& & & & \\ \hline
\textbf{Rank-20}& & & & \\ \hline
\end{tabular}
\caption{Compression test of OSNet on 25 pairs from Market-1501 dataset.}
\label{tab:compressiontest_result}
\end{table}

\begin{table}[H]
\centering
\begin{tabular}{|>{\centering\arraybackslash}
    p{0.17\linewidth} 
    |>{\centering\arraybackslash}p{0.17\linewidth} 
    |}
\hline
\multicolumn{2}{|c|}{\cellcolor[HTML]{D8E9F7}\textbf{Test on iLIDS-VID}}\\ \hline
\textbf{mAP}    & \\ \hline
\textbf{Rank-1} & \\ \hline
\textbf{Rank-5} & \\ \hline
\textbf{Rank-10}& \\ \hline
\textbf{Rank-20}& \\ \hline
\end{tabular}
\caption{Compression test of OSNet on 25 pairs from Market-1501 dataset.}
\label{tab:compressiontest_result}
\end{table}


Based on the results presented in Section \ref{sec: robustosnet}, the problem of images being out of distribution is not adequately solved through data augmentation alone. Therefore, a second experiment was conducted where the baseline model presented in Section \ref{sec: baselineosnet} was combined with a \ac{SR} model. Before evaluating the sequential pipeline, the \ac{SR} module itself was trained and validated to ensure adequate reconstruction quality.

\subsection{EDSR Training}
Two training strategies were explored for the \ac{EDSR} model: training from scratch on RELLISUR data and finetuning a model pretrained on \ac{DIV2K}. Both models used the baseline configuration described in Section \ref{sec:edsr_implementation}.

\subsubsection{Training from Scratch}
The model trained from scratch on cropped RELLISUR data showed a typical learning progression. Training loss started at approximately 0.12 and decreased rapidly to around 0.095 within the first 25 epochs. The loss continued to decrease gradually and stabilized around 0.089 after epoch 150. Validation \ac{PSNR} started at 23.7 dB and improved to approximately 25.3 dB within the first 25 epochs. The highest validation \ac{PSNR} of 25.72 dB was achieved at epoch 113. After this point, validation \ac{PSNR} fluctuated between 25.3 and 25.7 dB without further improvement, indicating that the model had converged.

\subsubsection{Finetuning from DIV2K}
The model pretrained on \ac{DIV2K} and finetuned on cropped RELLISUR data exhibited a different training pattern. Training loss started at approximately 0.078 and decreased only marginally to around 0.073 over the course of training. This low initial loss indicated that the pretrained weights already captured general \ac{SR} features. The highest validation \ac{PSNR} of 25.49 dB was achieved at epoch 1, after which validation \ac{PSNR} gradually declined to approximately 25.20 dB. This behavior suggested that continued training on the domain-specific data caused slight overfitting to the training distribution. A reduced learning rate of $5 \times 10^{-6}$ was used to prevent catastrophic forgetting of the pretrained features.

\subsubsection{Comparison of Training Strategies}
Figure \ref{fig:edsr_comparison} compares the training curves of both \ac{EDSR} models. The scratch model achieved higher validation \ac{PSNR} (25.72 dB versus 25.49 dB) but the finetuned model maintained lower training loss throughout. This apparent contradiction is explained by the different training distributions: the scratch model specialized heavily on the cropped validation images, while the finetuned model retained broader generalization capabilities from \ac{DIV2K} pretraining. The implications of this difference became evident when evaluating on the out-of-distribution test set, as presented in Chapter \ref{cha: results}.

\begin{figure}[H]
    \centering
    \includegraphics[width=\textwidth]{figures/images/cropped_models_comparison.png}
    \caption{Comparison of training curves for \ac{EDSR} trained from scratch (blue) and \ac{EDSR} pretrained on \ac{DIV2K} and finetuned (green). Top left: Training loss. Top right: Validation loss. Bottom left: Validation \ac{PSNR}. Bottom right: Learning rate schedule.}
    \label{fig:edsr_comparison}
\end{figure}

\subsection{SwinIR Baseline}
In addition to the trained \ac{EDSR} models, three pretrained SwinIR models were evaluated without finetuning to establish a baseline comparison with a state-of-the-art transformer-based \ac{SR} architecture. SwinIR Classic and SwinIR Real (Medium) were trained for classical image \ac{SR}, while SwinIR Real (Large) was trained with complex degradation models intended for real-world artifacts. These models were evaluated directly on the RELLISUR test set to assess their zero-shot generalization capabilities.

\subsection{Sequential Evaluation Setup}
The sequential evaluation pipeline passed images through the trained \ac{SR} model before feeding the upscaled output to the \ac{ReID} model. This setup allowed direct comparison between the baseline \ac{ReID} performance and the \ac{SR}-enhanced pipeline under identical test conditions. The evaluation was conducted on the OOD test conditions described in Table \ref{tab:newood_conditions}.



\section{Joint EDSR and Baseline OSNet}
The results from \ref{sec: sr+baseline} indicates that sequential training does not provide a satisfactory result. The final experiment is conducted by training the OSNet baseline model with EDSR in a joint end-to-end framework. 

\begin{table}[H]
\centering
\begin{tabular}{|
    >{\centering\arraybackslash}p{0.17\linewidth} |
    >{\centering\arraybackslash}p{0.17\linewidth} |
    >{\centering\arraybackslash}p{0.17\linewidth} |
    >{\centering\arraybackslash}p{0.17\linewidth} |
    >{\centering\arraybackslash}p{0.17\linewidth} |
    }
\hline
\multicolumn{5}{|c|}{\cellcolor[HTML]{D8E9F7}\textbf{Test on Different Scales}} \\\hline
\rowcolor[HTML]{D8E9F7}
{} & Scale 1& Scale 0.75& Scale 0.5&
Scale 0.25\\ \hline
\textbf{mAP}    & & & & \\ \hline
\textbf{Rank-1} & & & & \\ \hline
\textbf{Rank-5} & & & & \\ \hline
\textbf{Rank-10}& & & & \\ \hline
\textbf{Rank-20}& & & & \\ \hline
\end{tabular}
\caption{Compression test of OSNet on 25 pairs from Market-1501 dataset.}
\label{tab:compressiontest_result}
\end{table}

\begin{table}[H]
\centering
\begin{tabular}{|
    >{\centering\arraybackslash}p{0.17\linewidth} |
    >{\centering\arraybackslash}p{0.17\linewidth} |
    >{\centering\arraybackslash}p{0.17\linewidth} |
    >{\centering\arraybackslash}p{0.17\linewidth} |
    >{\centering\arraybackslash}p{0.17\linewidth} |
    }
\hline
\multicolumn{5}{|c|}{\cellcolor[HTML]{D8E9F7}\textbf{Test on Compression with Differing Quality Levels}} \\\hline
\rowcolor[HTML]{D8E9F7}
{} & 100 & Mild 75& Medium 50&
Heavy 25\\ \hline
\textbf{mAP}    & & & & \\ \hline
\textbf{Rank-1} & & & & \\ \hline
\textbf{Rank-5} & & & & \\ \hline
\textbf{Rank-10}& & & & \\ \hline
\textbf{Rank-20}& & & & \\ \hline
\end{tabular}
\caption{Compression test of OSNet on 25 pairs from Market-1501 dataset.}
\label{tab:compressiontest_result}
\end{table}

\begin{table}[H]
\centering
\begin{tabular}{|>{\centering\arraybackslash}
    p{0.17\linewidth} 
    |>{\centering\arraybackslash}p{0.17\linewidth} 
    |}
\hline
\multicolumn{2}{|c|}{\cellcolor[HTML]{D8E9F7}\textbf{Test on iLIDS-VID}}\\ \hline
\textbf{mAP}    & \\ \hline
\textbf{Rank-1} & \\ \hline
\textbf{Rank-5} & \\ \hline
\textbf{Rank-10}& \\ \hline
\textbf{Rank-20}& \\ \hline
\end{tabular}
\caption{Compression test of OSNet on 25 pairs from Market-1501 dataset.}
\label{tab:compressiontest_result}
\end{table}
