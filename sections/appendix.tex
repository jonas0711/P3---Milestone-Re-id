\chapter{Appendix}\label{cha:appendix}

\subsection{Use of AI}
In carrying out the project, both in developing the system and in documenting the implementation and results, generative AI has been used. Multiple large language models, LLMs have been used for various aspects during the project, including ChatGPT, Claude, Perplexity, Gemini and CoPilot. The use of a specific model depends on individual preferences and applications.
\\
During documentation of the project, generative AI have been useful in correcting language, ensuring that the message come across clearly. LLMs is excellent at repetitive task such as formatting data. When applied to such tasks, the outputs have been reviewed afterward to make sure nothing is altered. Furthermore, LLMs have been used when setting up layout in LaTeX as well as correcting LaTeX errors. Finally, LLM has been used to summarize and presenting methods, giving a preunderstanding to a subject before further reading.
\\
In developing the AI system, generative AI have mostly been used for debugging. Some error messages are difficult to decode. Generative AI can not only simplify the problem but also assist with code changes to solve the problem. This drastically reduces the time spend on debugging, allowing for time spend on improving model and testing different configurations. In addition, Generative AI has been used for a quick documentation description, examples of application of methods and brainstorming alternative solutions. 
