\chapter{Problem Analysis}
\label{cha:problemanalysis}
Chapter introduction

\section{Surveillance and Surveillance Cameras: Private and Police Usage}

Camera surveillance has become a common and visible part of Danish society in the 21st century. According to estimates from SikkerhedsBranchen, around 1.5 million surveillance cameras watch over citizens' daily activities throughout Denmark \cite{overvagningsekspert2025}. This large number of cameras creates tension between their use as security and crime prevention tools on one hand, and their potential impact on privacy rights on the other \cite{videnskab2024}.

\subsection{Police Application of Surveillance Cameras}
\label{subsec:Police_Application}

Police use camera surveillance for two main purposes: preventing crime and investigating crimes that have already happened. Since 2018, Danish police have installed 450 "security cameras" in public places as part of the government's security and safety package \cite{overvagningsekspert2025,videnskab2024}. These cameras are placed strategically to monitor public areas during events that might threaten public safety, such as demonstrations, high-risk football matches, and busy nightlife areas.
\\\\
Investigating serious crimes is the most important use of these cameras. As Police Inspector Johannes Sølvsten Jönsson explains: "Materiale fra overvågningskameraer er ofte helt centralt i vores efterforskning af blandt andet grove forbrydelser som voldtægt og drab" (translated: "Material from surveillance cameras is often very important to our investigation of serious crimes such as rape and murder") \cite{politiet2024opfordrer}. Police also use other camera technologies including body-worn cameras (bodycams), which help reduce conflicts, provide strong evidence, and give objective records of police behavior. Additionally, drones help police understand complex situations during search operations and when analyzing large crime scenes.
\\\\
An important part of police surveillance strategy is \ac{POLCAM}, which works as a nationwide system for finding locations and contact information for both private and public cameras \cite{politiet2024registrer,sikkerhedsbranchen2024,securitas2024guide}. The register does not give police direct access to cameras, but allows them to quickly identify relevant cameras along suspected escape routes after crimes happen \cite{politiet2024opfordrer}. This system shows how Danish law enforcement has effectively integrated privately owned surveillance systems into its investigation tools without bearing the costs of installation and maintenance.

\subsection{Private and Commercial Motivations for Surveillance}

Private citizens install surveillance cameras mainly to protect their property from crimes like burglary and vandalism \cite{bolius2024}. The main reason is wanting to discourage potential criminals through visible cameras and warning signs \cite{bolius2024}. Also, feeling safer plays an important role and being able to see live images from home through smartphone apps gives people a sense of control and peace of mind, no matter where they are \cite{bolius2024}.
\\\\
For retail businesses and other companies, video surveillance is a critical business tool that serves several purposes. The main goal is preventing and investigating economic crimes, including shoplifting, employee theft, robbery, and vandalism \cite{bolius2024}. In 2024, nearly 28,000 shoplifting incidents were reported, which was an increase of about 15\% from 2023 \cite{retailnews2025,danskerhverv2025rapport}. Eighty percent of Danish stores have experienced more theft incidents, making surveillance necessary for retail operations \cite{securityuser2025,danskerhverv2025rapport}. Beyond crime prevention, surveillance is used for business analysis of customer behavior and optimizing staff resources \cite{securitas2024butik}.

\subsection{Preventive Effects}
\label{subsec:preventive_effects}

The idea behind surveillance as a crime prevention tool is based on rational choice theory, where visible cameras should increase the perceived risk of being caught and therefore stop potential crimes from being committing crimes. Surveillance is deeply embedded in political arguments, as shown in the government's "Security Package" \cite{tryghedspakke2019}.
\\\\
Research evidence for this preventive effect is mixed. The Danish Crime Prevention Council concludes that surveillance shows clear preventive effects only in very specific situations, mainly for property crimes in limited areas \cite{dkr_tv_overvaagning}. Research broadly agrees that surveillance has little to no preventive effect on violent crimes like assault and robbery, because these crimes often happen on impulse without careful planning \cite{videnskab2024}. This does not mean surveillance cameras lack value, rather their primary documented value lies in their use as investigation tools after crimes have occurred.
\\\\
In contrast, there is clear agreement about the value of video recordings as investigation tools. Recordings help police identify criminals, understand how incidents happened, clear innocent people, and provide objective evidence. Danish criminal cases demonstrate this value: in the Emilie Meng murder case from 2016, surveillance footage from Korsør Station, though of poor quality, was enhanced by specialists to identify a suspect's vehicle, which combined with telecom data led to an arrest \cite{emilie_meng_avisen,emilie_meng_wiki}. Similarly, in the 2022 Mia Skadhauge Stevn case, surveillance cameras captured the victim entering a vehicle on Vesterbro in Aalborg, and subsequent camera footage allowed police to read the license plate, leading to a rapid breakthrough in the investigation \cite{mia_stevn_avisen}. These cases illustrate how systematic analysis of surveillance material can be crucial for investigation breakthroughs.

\subsection{Ethical Considerations}

With the number of surveillance cameras increasing, so does the risk of invading the privacy of the public \cite{overvagningsekspert2025}. Private parties are not permitted to monitor areas used for public access \cite{lov_tv_overvaagning2023} and surveillance cameras that conduct video surveillance of such an area must be registered in the \acs{POLCAM} within 14 days of installation \cite{politiet2024registrer}. 
\\\\
These registration requirements serve multiple privacy-protecting functions. First, they create transparency about where surveillance occurs, allowing citizens to be aware of monitored areas. Second, registration enables oversight authorities to verify that cameras are placed and used according to legal requirements. Third, the obligation to register discourages unauthorized surveillance of public spaces, as unregistered cameras violating the placement rules can be identified and owners held accountable \cite{kromann_polcam2021}. Together, these mechanisms help ensure that the use of cameras in public spaces remains controlled and legally compliant, thereby limiting the risk of unauthorized collection of image and video materials.
\\\\
However, new technologies such as data analysis, facial recognition and risk assessment algorithms make it difficult to preserve the right to privacy \cite{overvagningsekspert2025}. A fundamental principle in democratic surveillance is proportionality: surveillance should target specific criminal activities or individuals, not monitor the entire population to catch a few offenders. Systems that enable continuous monitoring of all citizens violate this principle and infringe upon the right to privacy, as they subject innocent people to surveillance without reasonable suspicion \cite{menneskeret_overvaagning,justitia_overvaagning}. As described in Section \ref{subsec:preventive_effects}, these new technologies are valuable tools for law enforcement. Yet, if people are under the impression that they are under constant surveillance, it will inevitably change their behavior in their everyday life \cite{overvagningsekspert2025}. This raises the problem of balancing the value of surveillance technologies for law enforcement against the risk it poses to freedom of expression through self-censorship \cite{overvagningsekspert2025}.

\section{Milestone Systems}

Milestone Systems is a global provider of video surveillance software, headquartered in Brøndby, Denmark. Its product portfolio centers on the open-platform XProtect VMS alongside the BriefCam analytics platform and Arcules VSaaS. XProtect is trusted in 500,000+ installations worldwide and supports more than 14,000 devices, reflecting Milestone’s focus on scale and broad hardware interoperability. In public-safety contexts including law enforcement Milestone positions XProtect as the operational backbone for situational awareness, incident response, and evidence handling. BriefCam integrates tightly with XProtect and is presented by Milestone as the advanced video-analytics layer for rapid video review, real-time alerts, and deeper research/forensics (the “Review, Respond, Research” modules), with official installation and integration guidance available from Milestone’s documentation. Altogether, the Milestone BriefCam stack is widely referenced for law-enforcement workflows that compress hours of footage into actionable leads and speed up investigative timelines \cite{MilestoneSystems2025}.

\subsection{BriefCam}
As mentioned above, BriefCam is presented by Milestone as the advanced video-analytics layer tightly integrated with XProtect. In law-enforcement and public-safety workflows, BriefCam supports rapid video review, real-time alerting, and deeper investigative research through its Review, Respond, and Research modules. Of particular relevance to this report, BriefCam provides person re-identification (Re-ID) via two complementary approaches: (i) face-based matching using operator-validated watchlists, and (ii) appearance-similarity search that links visually similar individuals across non-overlapping cameras based on visual features and attributes (e.g., clothing color or carried objects). The official XProtect integration embeds these capabilities directly in the Smart Client and ties alerts and evidence handling back into XProtect’s unified management, helping compress hours of footage into actionable leads and accelerating investigative timelines \cite{MilestoneSystems2025}.

