\chapter{Problem Analysis}
\label{cha:problemanalysis}
This project originates from a project proposal by the company Milestone Systems, with the objective of improving a person \acf{ReID} system in cases of low-resolution images, proposing integration with \acf{SR}. This forms the foundation of this project.
\\\\
Before designing and implementing a solution, it is important to understand the problem of person \ac{ReID} when working with surveillance footage. This chapter will present the motivation behind video surveillance, then a short introduction to the task of \ac{ReID} and at last the ethical considerations regarding surveillance and \ac{ReID}. 

\section{Video Surveillance} \label{sec:videosurveillanceprobana}
% Mangler en pointe omkring kvaliteten af billederne
%1.5 Fordi man vil dække større områder for crime, så skal man enten hænge mange kameraer op eller et kamera skal dække et større område
%2. Kameraerne dækker stort område = dårlig kvalitet + ikke mega hd - kommer til at fylde meget - i forvejen mange timer = komprimeret
%Quality of the camera - how many pixels do they have (?)
%Quality of the video - compression of the clips (?)

Video surveillance is widely used today, with an estimated 1.5 million surveillance cameras deployed in danish society, serving as both a preventive measure and an investigative tool for solving crimes \cite{overvagningsekspert2025}.
Private citizens, retail businesses, and other companies use video surveillance to prevent and capture evidence of crimes such as robbery and vandalism. While visible cameras and warning signs are used to discourage potential criminals, law enforcement can use the footage as evidence in the investigations \cite{bolius2024}. 
\\\\
Since 2018, Danish police have installed 450 surveillance cameras in public places as part of the government's security and safety package \cite{overvagningsekspert2025,videnskab2024}. These cameras are placed strategically to monitor public areas during events that might threaten public safety, such as demonstrations, high-risk football matches, and busy nightlife areas \cite{dyhrberg2024natteliv}. 
\\
Besides having their own cameras, law enforcement has a nationwide system \ac{POLCAM} for finding locations and contact information on both private and public cameras \cite{politiet2024registrer,sikkerhedsbranchen2024,securitas2024guide}. The register does not give police direct access to the footage, but allows quick identification of cameras relevant to an ongoing investigation \cite{politiet2024opfordrer}. This approach allows police to benefit from privately owned cameras, adding supplementary material to investigations. 
\\\\
Material from surveillance cameras is a critical part of law enforcement investigations of serious crimes such as rape and murder \cite{politiet2024opfordrer}.
\\\\
The idea behind surveillance as a crime prevention tool is based on rational choice theory, where visible cameras should increase the perceived risk of being caught and therefore stop potential crimes from being committed \cite{tryghedspakke2019}. However, research evidence for this preventive effect is mixed. The Danish Crime Prevention Council concludes that surveillance shows clear preventive effects only in very specific situations, mainly for property crimes in limited areas \cite{dkr_tv_overvaagning}. Research broadly agrees that surveillance has little to no preventive effect on violent crimes like assault and robbery, because these crimes often happen on impulse without careful planning \cite{videnskab2024}. In contrast, there is clear agreement about the value of video recordings as investigation tools. Recordings help police understand how incidents happened, clear innocent people, provide objective evidence, identify criminals and map their movements. 

\section{Person Re-Identification}\label{ReID-prob}

Person \acf{ReID} is the task of identifying the same person across multiple non-overlapping camera views. Unlike related applications such as facial recognition, or tracking movement from a single camera, person \acs{ReID} aims to operate across larger areas that extend beyond the \ac{FOV} of a single camera. This makes surveillance a great source of data for this task.
\\\\
The algorithm behind person \ac{ReID} must function across multiple non-overlapping camera views, the key challenges are adapting to variations in camera angles, resolution, lighting, background, and partial or full occlusion of the \ac{POI} \cite{IntroPReID2023}. Furthermore, due to the nature of the task, each person inhabits a unique class that is rarely observed more than once, causing \ac{ReID} to inherently pose a \ac{FSL} problem. The main challenge then becomes designing a pipeline which extracts robust and discriminative features across diverse environments, while also ensuring enough information is kept for \ac{FSL} to be viable. The technical aspects, as well as detailed explanations of the challenges will be covered in Section \ref{sec:ReID-tech}.

\section{Ethical Considerations}
With the number of surveillance cameras increasing, so does the risk of invading the privacy of the public \cite{overvagningsekspert2025}. Danish law requires private parties to register surveillance cameras that monitor public areas in \ac{POLCAM} within 14 days of installation \cite{lov_tv_overvaagning2023, politiet2024registrer}. These registration requirements serve as multiple privacy-protecting functions, by creating transparency for the public about where surveillance occurs, and provide authorities with an overview of compliance with the law. However, this does not guarantee that all surveillance activities are in fact registered or compliant.
\\\\
New technologies such as data analysis, facial recognition and risk assessment algorithms further endangers the right to privacy \cite{overvagningsekspert2025}. Democratic surveillance should target specific criminal activities or individuals, not monitor the entire population to catch a few offenders. Systems that enable continuous monitoring of all citizens violate the right to privacy, as they subject innocent people to surveillance without reasonable suspicion \cite{menneskeret_overvaagning,justitia_overvaagning}.
If people believe they are under constant surveillance, it will likely change their behavior in everyday life \cite{overvagningsekspert2025}. This raises the problem of balancing the value of surveillance technologies for law enforcement against the risk it poses to freedom of expression through self-censorship \cite{overvagningsekspert2025}. For this reason, a person \ac{ReID} system should not be used for general identification of all individuals, but only in investigations of serious crimes.
\\\\
In recent years, surveillance footage and photo enhancement techniques have been used to solve murders both in Denmark and internationally \cite{overvagningsekspert2025}. In the Emilie Meng murder case from 2016, surveillance footage from Korsør Station was enhanced by specialists, despite its poor quality, enabling identification of a suspect's vehicle. Combined with telecom data, this led to an arrest \cite{emilie_meng_avisen,emilie_meng_wiki}. Similarly, in the 2022 Mia Skadhauge Stevn case, surveillance cameras captured the victim entering a vehicle on Vesterbro in Aalborg, and subsequent camera footage allowed police to read the license plate, leading to a rapid breakthrough in the investigation \cite{mia_stevn_avisen}. These cases illustrate how systematic analysis of surveillance material can be crucial for investigation breakthroughs.
\\\\
As many other technologies, person \ac{ReID} can be used in beneficial and harmful ways. As mentioned previously, it has been used to track and identify criminals and victims. However, in more extreme applications, the same technology could enable large-scale surveillance systems capable of tracking individuals continuously. 
\\\\
In 2024, the police in the United Kingdom started deploying live face recognition using mobile cameras, and in 2025 fixed cameras to capture images of people walking by, matching their face against a database of known offenders. According to Metropolitan Police, the system has brought 450 arrests since January 2025, but also resulted in seven false identifications in the same time period \cite{BBC-london-facerecognition}. Getting falsely accused of crimes not committed based on a match in surveillance footage is problematic, it is a waste of resources, and it creates mistrust in the system.
\\\\
Another controversial application is face recognition search engines such as PimEyes, which allow users to search for a persons face on the open internet, solely based on a reference photo. In 2022, PimEyes faced criticism due to concerns that the tool could be used for stalking and other types of harassment \cite{pimeyes-critic-bigbrother, pimeyes-critic-BBC}. At present, PimEyes claims to not have access to social media sites such as Instagram and Facebook \cite{pimeyes-homepage}. A restriction that prevents tracking individuals even more online \cite{pimeyes-critic-BBC}.
\\\\
Even though a technology like person \ac{ReID} poses  risk to citizens' privacy, the positive applications are significant. This project focuses on improving edge case scenarios where poor resolution makes identification difficult, which is particularly relevant in law enforcement investigations. The ethical considerations surrounding such technologies are recognized, however, not discussed further in this project.

\section{Summary}
In summary of this chapter, the motivation behind the use of surveillance cameras lies in both the preventative and investigative qualities. The footage is a valuable tool for law-enforcement when investing crimes, especially regarding identifying the \ac{POI}, which can be done through person \ac{ReID}. The main challenge is designing a pipeline that can extract robust and discriminative features across the different environments, which will be explored further in Section \ref{sec:ReID-tech}. Working with surveillance footage, ethical considerations will constrain the deployment of a person \ac{ReID} system in order to protect the right to privacy.