\chapter{Introduction}
\label{cha:introduction}

Video surveillance has become a fundamental component of modern law enforcement. In Denmark, an estimated 1.5 million surveillance cameras form an extensive network that police increasingly rely upon for criminal investigations \cite{overvagningsekspert2025}. High-profile cases, such as the Emilie Meng murder investigation and the Mia Skadhauge Stevn case, demonstrate how surveillance footage can provide crucial evidence leading to breakthroughs and successful prosecutions \cite{emilie_meng_avisen, mia_stevn_avisen}.
\\\\
However, the effectiveness of surveillance systems is constrained by technical limitations. Footage frequently suffers from low resolution due to camera placement at considerable distances, hardware limitations, and challenging environmental conditions such as poor lighting and adverse weather \cite{arxiv_suerres2021}. This limits the investigative value of recorded footage despite its abundance. The challenge of extracting usable information from degraded surveillance imagery represents a significant practical problem for law enforcement agencies.
\\\\
The intersection of artificial intelligence and surveillance technology presents an opportunity to address this challenge. \ac{SR} techniques, which reconstruct higher-resolution images from low-resolution observations, have advanced significantly through deep learning. When combined with person \ac{ReID} systems that track individuals across multiple camera views, these technologies offer potential to enhance investigative capabilities. This convergence of computer vision techniques with real-world law enforcement needs motivated the focus of this project.
\\\\
The proliferation of surveillance technologies also raises ethical considerations regarding privacy and civil liberties. Danish regulations, including mandatory registration through the \ac{POLCAM} and restrictions on monitoring public spaces, reflect ongoing efforts to balance security needs against fundamental rights to privacy \cite{politiet2024registrer, lov_tv_overvaagning2023}. As \ac{AI} and computer vision advance, these ethical tensions intensify, necessitating careful consideration of proportionality in system design \cite{overvagningsekspert2025, menneskeret_overvaagning}.
\\\\
This project investigates whether \ac{SR} preprocessing can improve the accuracy of person re-identification networks when applied to low-resolution surveillance imagery. The scope is deliberately focused: the project evaluates super-resolution as a preprocessing step for an existing re-identification model, measuring performance improvements on degraded surveillance footage. The project does not develop novel super-resolution or re-identification architectures, nor does it implement a complete operational system with user interfaces for deployment. Instead, the emphasis is on empirical evaluation of whether this technical approach offers practical value for law enforcement applications. Milestone Systems, a Danish surveillance software provider headquartered in Brøndby, serves as a reference point for understanding contemporary surveillance workflows, though system integration with commercial platforms falls outside the project scope \cite{MilestoneSystems2025}.
\\\\
