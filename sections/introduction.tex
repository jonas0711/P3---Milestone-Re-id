\chapter{Introduction}
\label{cha:introduction}

Video surveillance has become an essential tool for both crime prevention and investigation in modern society. With the increasing availability of surveillance footage from both public and private sources, law enforcement agencies face a growing challenge: efficiently identifying and tracking individuals across multiple camera feeds. Person \ac{ReID} addresses this challenge by enabling the automatic identification of the same individual across non-overlapping camera views. However, the practical effectiveness of \ac{ReID} systems is often limited by the quality of surveillance footage, particularly when subjects appear at a distance from the camera or when footage has been compressed to reduce storage requirements.

This project originated from a collaboration with Milestone Systems, a leading provider of video surveillance software, who proposed investigating the integration of \ac{SR} techniques into \ac{ReID} pipelines. The central hypothesis was that enhancing \ac{LR} images through \ac{SR} before passing them to a \ac{ReID} network could improve identification accuracy. This report documents the investigation, design, implementation, and evaluation of such an integrated system.

\section{Milestone Systems}\label{sec:milestone}
Milestone Systems is a global provider of video surveillance software, headquartered in Brøndby, Denmark. The company portfolio consists of the video management software XProtect VMS, the video surveillance service Arcules, and the analytics platform BriefCam. BriefCam integrates seamlessly with XProtect and is presented by Milestone as the advanced video-analytics layer for rapid video review, real-time alerts, and deeper research or forensics \cite{MilestoneSystems2025}. Of particular relevance to this report, BriefCam provides person \ac{ReID} via two complementary approaches, being face-based matching using operator-validated watchlists and appearance-similarity search that links visually similar individuals across non-overlapping cameras based on visual features and attributes \cite{MilestoneSystems2025}.

\section{Report Overview}\label{sec:reportoverview}
This report is structured to guide the reader from problem understanding through technical analysis to implementation and evaluation. Chapter \ref{cha:problemanalysis} presents the problem analysis, examining the context of video surveillance, the task of person \ac{ReID}, and the ethical considerations that must guide system design. Chapter \ref{cha: technicalanalysis} provides the technical analysis, covering surveillance equipment limitations, computer vision fundamentals, the technical aspects of person \ac{ReID}, \ac{SR} techniques, and related works combining these fields. Based on these analyses, Chapter \ref{cha: problemstatement} formulates the problem statement that guides the project. Chapter \ref{cha:reqspec} defines the requirement specification derived from the preceding analyses.

The remaining chapters document the practical work. Chapter 6 presents the data analysis of the datasets used for training and evaluation. Chapter 7 describes the methods applied in designing and training the system components. Chapter 8 details the implementation of the \ac{SR} and \ac{ReID} modules. Chapter 9 describes the experiments conducted to evaluate the system. Chapter 10 presents the results, followed by Chapter 11 which discusses the findings and their implications. Finally, Chapter 12 concludes the report with a summary of contributions and suggestions for future work.