\chapter{Introduction}
\label{cha:introduction}

This project originates from a project proposal by the company Milestone Systems, with the objective of improving a person \acf{ReID} system in cases of low-resolution images, proposing integration with \acf{SR}. This forms the foundation of this project.
\\\\
Video surveillance is widely used today, with an estimated 1.5 million surveillance cameras deployed in Danish society, serving as both a preventive measure and an investigative tool for solving crimes \cite{overvagningsekspert2025}. Person \acs{ReID} addresses the need to identify and track individuals across multiple camera feeds by allowing the automatic identification of the same individual across non-overlapping camera views. However, how well \acs{ReID} systems work is often limited by the quality of surveillance footage, particularly when subjects appear at a distance from the camera or when footage has been compressed to reduce storage requirements, resulting in low-resolution images that make identification difficult.
\\\\
Based on this, it is evident that improving \acs{ReID} performance on low-resolution images should be a priority. This project focuses on improving edge case scenarios where poor resolution makes identification difficult, which is particularly relevant in law enforcement investigations. The main idea is that enhancing low-resolution images through \acs{SR} before passing them to a \acs{ReID} network could improve identification accuracy. As such, the focus of this report is to investigate the integration of \acs{SR} and \acs{ReID} systems to gain an understanding of what can be done to improve identification accuracy on low-resolution surveillance footage. Based on the investigation, potential solutions are proposed, implemented, evaluated, and discussed.

\section{Milestone Systems}\label{sec:milestone}
Milestone Systems is a global provider of video surveillance software, headquartered in Brøndby, Denmark. The company portfolio includes the video management software XProtect VMS, the video surveillance service Arcules, and the analytics platform BriefCam. BriefCam integrates seamlessly with XProtect and provides person \acs{ReID} via two complementary approaches: face-based matching using operator-validated watchlists and appearance-similarity search that links visually similar individuals across non-overlapping cameras based on visual features and attributes \cite{MilestoneSystems2025}. This collaboration provided the foundation for investigating how \ac{SR} techniques could improve \ac{ReID} performance on low-resolution surveillance footage.

